%Matrix Calculus
%HU, Pili
%Craete: 20120330

%Modify the relative path accordingly
%HU, Pili
%Create: 20120330
%Modify: 20120330
%The unified entry to include in my tutorial series

%HU, Pili
%Create: 20110910
%Modify: 20120330
%purpose of this file is to gather commonly used
%mathematical abbreviations, to speed up writing
%notes

\documentclass[11pt,a4paper]{article}
\usepackage[utf8x]{inputenc}
\usepackage{ucs}
\usepackage{amsmath}
\usepackage{amsfonts}
\usepackage{amssymb}
\usepackage{amsthm}
\usepackage{url}
\usepackage{graphicx}

\usepackage{fancyhdr}
\pagestyle{fancy}
\fancyhead{}

%=====Calculus======
%the following commands are not originated by me
%I pick them from http://www-solar.mcs.st-and.ac.uk/~clare/Latex/
%the following line controls the style of patial derivative
%1), use \dfrac, height is larger, looks good. 
%2), use \frac, also work, but space looks limited. 
\newcommand{\myfrac}[2]{\dfrac{#1}{#2}}
\newcommand{\diff}[2]{\myfrac{{\rm d}#1}{{\rm d}#2}}
\newcommand{\ndiff}[3]{\myfrac{{\rm d}^{#3}#1}{{\rm d}#2^{#3}}}
\newcommand{\pdiff}[2]{\myfrac{\partial #1}{\partial #2}}
\newcommand{\npdiff}[3]{\myfrac{\partial^{#3} #1}{\partial #2^{#3}}}
\newcommand{\e}[1]{\ensuremath{{\rm e}^{#1}}}
\newcommand{\ldiff}[2]{\ensuremath{{\rm d}#1/{\rm d}#2}}
\newcommand{\lpdiff}[2]{\ensuremath{\partial#1/\partial#2}}
\newcommand{\lnpdiff}[3]{\ensuremath{\partial^{#3}#1/\partial#2^{#3}}}
\newcommand{\dif}[1]{\mathrm{d}#1}

%20120330
%The reason I don't copy the original file as a whole
%is that it contains too many individually preferred 
%definitions. 
%
%I start with those basic symbols and adapt them in use. 

%=====Matrix======
\newcommand{\tr}[1]{\mathrm{Tr}\left[#1\right]}
\newcommand{\tran}[1]{#1^\mathrm{T}}
%The following shorthand of matrix may be convenient. 
%However, it is so short that I'm worried it may 
%collide with something else. I don't use at present.
%\newcommand{\m}[1]{\mathbf{#1}}
\newcommand{\adj}[0]{\mathrm{adj}}

%=====Theorem definitions=====
\newcounter{mytheoremorder}
\newtheorem{mydef}{Definition}
\newtheorem{myaxm}{Axiom}
\newtheorem{mythm}[mytheoremorder]{Theorem}
\newtheorem{myprop}[mytheoremorder]{Proposition}
\newtheorem{myex}{Example}

%=====Optimization====
\DeclareMathOperator*{\argmax}{arg\,max}
\DeclareMathOperator*{\argmin}{arg\,min}
\newcommand{\maximize}[0]{\mathrm{Maximize~}}
\newcommand{\minimize}[0]{\mathrm{Minimize~}}

%=====Probability====
\newcommand{\E}[0]{\mathbb{E}}
\newcommand{\var}[0]{\mathrm{Var}}
\newcommand{\cov}[0]{\mathrm{Cov}}

%=====Quick and Unified Reference====
%20120505
%Usage: \eq{\ref{xxx}}
%The reason I keep "\ref" away from definition, 
%and let user type it every time is that: 
%currently I'm working on Texmaker, and it can 
%trigger a selection panel when the sequence 
%"\ref" is found. Maybe further configuration of 
%texmake can make it do the same thing when the 
%following self-defined sequence is detected.
%This is left for future work. 
\newcommand{\req}[1]{\textbf{Eq~{#1}}}
\newcommand{\rfig}[1]{\textbf{Fig~{#1}}}
\newcommand{\rtbl}[1]{\textbf{Tbl~{#1}}}
\newcommand{\rpg}[1]{\textbf{P~{#1}}}
\newcommand{\rsec}[1]{\textbf{Section~{#1}}}


%This usually doesn't need modification 
\author{HU, Pili\thanks{hupili [at] ie [dot] cuhk [dot] edu [dot] hk}}

%Modify them accordingly===
\title{Matrix Calculus: \\ Derivation and Simple Application}
\date{March 30, 2012\thanks{Last compile:\today}}

\begin{document}

\maketitle
%>============================================
\begin{abstract}
	Matrix Calculus\cite{wiki_mc} is a very useful tool in many 
	engineering problems. Basic rules of matrix calculus are 
	nothing more than ordinary calculus rules covered in 
	undergraduate courses. However, using matrix calculus, 
	the derivation process is more compact. This document is 
	adapted from the notes of a course the author recently attends.
	It builds matrix calculus from scratch. Only prerequisites 
	are basic calculus notions and linear algebra operation.  
	To get a quick executive guide, please refer to the cheat 
	sheet in the end. 
\end{abstract}
%<=======Abstract ENd=========================

%>============================================
\pagebreak
\tableofcontents
\pagebreak
%<=======TOC ENd==============================



\section{Introductory Example}

We start with an one variable linear function:
\begin{equation}
	f(x) = ax
\end{equation}

To be coherent, we abuse the partial derivative notation:
\begin{equation}
	\pdiff{f}{x} = a
	\label{eq:fax-single}
\end{equation}

Extending this function to be multivariate, we have:
\begin{equation}
	f(x) = \sum_{i}{a_ix_i} = \tran{a}x
\end{equation}
%The followings are the suggested transpose online? 
%What do you prefer? 
%I choose \mathrm at present. It looks reasonably good. 
%	a^\intercal
%	a^\mathsf{T}
%	a^\mathrm{T}
%	a^\top
%	a^\bot
Where $a = \tran{[a_1,a_2,\ldots,a_n]}$ and 
$x = \tran{[x_1,x_2,\ldots,x_n]}$. 
We first compute partial derivatives directly:
\begin{equation}
	\pdiff{f}{x_k} = \pdiff{(\sum_{i}{a_ix_i})}{x_k} = a_k 
\end{equation}
for all $k=1,2, \ldots, n$. Then we organize $n$ partial derivatives
in the following way:
\begin{equation}
	\pdiff{f}{x} = \left[
	\begin{matrix}
		\pdiff{f}{x_1} \\
		\pdiff{f}{x_2} \\
		\vdots \\
		\pdiff{f}{x_n}
	\end{matrix}
	\right]
	= \left[
	\begin{matrix}
		a_1 \\
		a_2 \\
		\vdots \\
		a_n
	\end{matrix}
	\right]
	= a
	\label{eq:fax-multi}
\end{equation}
The first equality is by proper definition and the rest roots from 
ordinary calculus rules. 

Eqn(\ref{eq:fax-multi}) is analogous to eqn(\ref{eq:fax-single}), except
the variable changes from a scalar to a vector. Thus we want to directly 
claim the result of eqn(\ref{eq:fax-multi}) without those intermediate steps 
solving for partial derivatives separately. Actually, we'll see soon 
that eqn(\ref{eq:fax-multi}) plays a core role in matrix calculus. 

Following sections are organized as follows:
\begin{itemize}
	\item Section(\ref{sec:derivation}) builds commonly used 
	matrix calculus rules from ordinary calculus and linear 
	algebra. Necessary and important properties of linear 
	algebra is also proved along the way. This section is not 
	organized afterhand. All results are proved when we need them. 
	\item Section(\ref{sec:application}) shows some applications 
	using matrix calculus. 
	\item Section(\ref{sec:cheat}) concludes a cheat sheet of 
	matrix calculus. Note that this cheat sheet may be different 
	from others. Users need to figure out some basic definitions 
	before applying the rules. 
\end{itemize}


\section{Derivation}
\label{sec:derivation}

\subsection{Organization of Elements}
From the introductary example, we already see that matrix calculus 
does not distinguish from ordinary calculus by fundamental rules. 
However, with better organization of elements and 
proving useful properties, we can simplify the derivation process 
in real problems. 

The author would like to adopt the following definition:
\begin{mydef}
	\label{def:org}
	For a scalar valued function $f(x)$, the result $\pdiff{f}{x}$
has the same size with $x$. 
\end{mydef}

In eqn(\ref{eq:fax-single}), $x$ is a 1-by-1 matrix and the result
$\pdiff{f}{x}=a$ is also a 1-by-1 matrix. In eqn(\ref{eq:fax-multi}), 
$x$ is a column vector(known as n-by-1 matrix) and the result
$\pdiff{f}{x}=a$ has the same size. 

\begin{myex}
\label{ex:tran_x}
By this definition, we have:
\begin{equation}
	\pdiff{f}{\tran{x}} = \tran{(\pdiff{f}{x})} = \tran{a}
\end{equation}
Note that we only use the organization definition in this example. 
Later we'll show that with some matrix properties, this formula 
can be derived without using $\pdiff{f}{x}$ as a bridge. 
\end{myex}

\subsection{Deal with Inner Product}

\begin{mythm}
	\label{thm:inner_product}
	If there's a multivariate scalar function $f(x) = \tran{a}x$, 
	we have $\pdiff{f}{x} = a$.
\end{mythm}

\begin{proof}
See introductary example. 
\end{proof}

Since $\tran{a}x$ is scalar, we can write it equivalently as the 
trace of its own. Thus, 
\begin{myprop}
	\label{prop:trace_inner_product}
	If there's a multivariate scalar function $f(x) = \tr{\tran{a}x}$, 
	we have $\pdiff{f}{x} = a$.
\end{myprop}

$\tr{\bullet}$ is the operator to sum up diagonal elements of a matrix. 
In the next section, we'll explore more properties of trace. 
As long as we can transform our target function into the form of 
theorem(\ref{thm:inner_product}) or proposition(\ref{prop:trace_inner_product}), 
the result can be written out directly. Notice in 
proposition(\ref{prop:trace_inner_product}), $a$ and $x$ are both vectors. 
We'll show later as long as their sizes agree, it holds for matrix $a$ and $x$. 

\subsection{Properties of Trace}

\begin{mydef}
Trace of square matrix is defined as: $\tr{A} = \sum_{i}{A_{ii}}$
\end{mydef}

\begin{mythm}
Matrix trace has the following properties:
\begin{itemize}
	\item (1) $\tr{A+B} = \tr{A} + \tr{B}$
	\item (2) $\tr{cA} = c \tr{A}$
	\item (3) $\tr{AB} = \tr{BA}$
	\item (4) $\tr{A_1A_2 \ldots A_n}=\tr{A_nA_1 \ldots A_{n-1}}$
	\item (5) $\tr{\tran{A}B} = \sum_{i}\sum_{j}{A_{ij}B_{ij}}$
	\item (6) $\tr{A} = \tr{\tran{A}}$
\end{itemize}
where $A,B$ are matrices with proper sizes, and $c$ is a scalar value. 
\end{mythm}

\begin{proof}
See wikipedia \cite{wiki_trace} for the proof. 
\end{proof}

Here we explain the intuitions behind each property
to make it easier to remenber. Property(1)
and property(2) shows the linearity of trace. 
Property(3) means two matrices' multiplication inside a 
the trace operator is commutative. Note that the matrix 
multiplication without trace is not commutative and 
the commutative property inside the trace does not hold 
for more than 2 matrices. Property (4) is the proposition of 
property (3) by considering $A_1A_2 \ldots A_{n-1}$ 
as a whole. It is known as cyclic property, so that you can 
rotate the matrices inside a trace operator. Property (5) 
shows a way to express the sum of element by element product using 
matrix product and trace. Note that inner product of two vectors 
is also the sum of element by element product. Property (5) 
resembles the vector inner product by form($\tran{A}B$). 
The author regards property (5) as the extension of inner product 
to matrices(Generalized Inner Product). 

\subsection{Deal with Generalized Inner Product}

\begin{mythm}
	\label{thm:gip}
	If there's a multivariate scalar function $f(x) = \tr{\tran{A}x}$, 
	we have $\pdiff{f}{x} = A$. ($A,x$ can be matrices). 
\end{mythm}

\begin{proof}
	Using property (5) of trace, we can write $f$ as:
	\begin{equation}
		f(x) = \tr{\tran{A}x} = \sum_{ij}A_{ij}x_{ij}
	\end{equation}
	It's easy to show:
	\begin{equation}
		\pdiff{f}{x_{ij}} = \pdiff{(\sum_{ij}A_{ij}x_{ij})}{x_{ij}} = A_{ij}
	\end{equation}
	Organize elements using definition(\ref{def:org}), it is proved. 
\end{proof}

With this theorem and properties of trace we revisit example(\ref{ex:tran_x}). 
\begin{myex}
	For vector $a,x$ and function $f(x) = \tran{a}x$
	\begin{eqnarray}
	 &&\pdiff{f}{\tran{x}} \\
		&=& \pdiff{( \tran{a}x )}{\tran{x}} \\
	\text{($f$ is scalar)}	&=& \pdiff{( \tr{\tran{a}x} )}{\tran{x}} \\
	\text{(property(3))}	&=& \pdiff{( \tr{x\tran{a}} )}{\tran{x}} \\
	\text{(property(6))}	&=& \pdiff{( \tr{a\tran{x}} )}{\tran{x}} \\
	\text{(property of transpose)}	&=& \pdiff{( \tr{\tran{(\tran{a})}\tran{x}} )}{\tran{x}} \\
	\text{(theorem(\ref{thm:gip}))}	&=& \tran{a}
	\end{eqnarray}
	The result is the same with example(\ref{ex:tran_x}), 
	where we used the basic definition. 
\end{myex}

The above example actually demonstrates the usual way of handling a 
matrix derivative problem. 

\section{Application}
\label{sec:application}

\subsection{The 2nd Induced Norm of Matrix}

\section{Cheat Sheet}
\label{sec:cheat}


%>============================================
\section*{Acknowledgements}
\addcontentsline{toc}{section}{Acknowledgements}
Thanks prof. XU, Lei's tutorial on matrix calculus. 
Besides, the author also benefit a lot from other online 
materials. 
%<=======Acknowledgements ENd=================

%>============================================
\addcontentsline{toc}{section}{References}
\input{../reference/gen_bib.bbl}
%<=======Bibliography ENd=====================

%>============================================
\section*{Appendix}
\addcontentsline{toc}{section}{Appendix}

%<=======Appendix ENd=========================

\end{document}
