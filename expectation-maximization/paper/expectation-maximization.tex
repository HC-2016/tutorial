%Expectation Maximization
%HU, Pili
%Craete: 20120415
%Update: 20120415

%Modify the relative path accordingly
%HU, Pili
%Create: 20120330
%Modify: 20120330
%The unified entry to include in my tutorial series

%HU, Pili
%Create: 20110910
%Modify: 20120330
%purpose of this file is to gather commonly used
%mathematical abbreviations, to speed up writing
%notes

\documentclass[11pt,a4paper]{article}
\usepackage[utf8x]{inputenc}
\usepackage{ucs}
\usepackage{amsmath}
\usepackage{amsfonts}
\usepackage{amssymb}
\usepackage{amsthm}
\usepackage{url}
\usepackage{graphicx}

\usepackage{fancyhdr}
\pagestyle{fancy}
\fancyhead{}

%=====Calculus======
%the following commands are not originated by me
%I pick them from http://www-solar.mcs.st-and.ac.uk/~clare/Latex/
%the following line controls the style of patial derivative
%1), use \dfrac, height is larger, looks good. 
%2), use \frac, also work, but space looks limited. 
\newcommand{\myfrac}[2]{\dfrac{#1}{#2}}
\newcommand{\diff}[2]{\myfrac{{\rm d}#1}{{\rm d}#2}}
\newcommand{\ndiff}[3]{\myfrac{{\rm d}^{#3}#1}{{\rm d}#2^{#3}}}
\newcommand{\pdiff}[2]{\myfrac{\partial #1}{\partial #2}}
\newcommand{\npdiff}[3]{\myfrac{\partial^{#3} #1}{\partial #2^{#3}}}
\newcommand{\e}[1]{\ensuremath{{\rm e}^{#1}}}
\newcommand{\ldiff}[2]{\ensuremath{{\rm d}#1/{\rm d}#2}}
\newcommand{\lpdiff}[2]{\ensuremath{\partial#1/\partial#2}}
\newcommand{\lnpdiff}[3]{\ensuremath{\partial^{#3}#1/\partial#2^{#3}}}
\newcommand{\dif}[1]{\mathrm{d}#1}

%20120330
%The reason I don't copy the original file as a whole
%is that it contains too many individually preferred 
%definitions. 
%
%I start with those basic symbols and adapt them in use. 

%=====Matrix======
\newcommand{\tr}[1]{\mathrm{Tr}\left[#1\right]}
\newcommand{\tran}[1]{#1^\mathrm{T}}
%The following shorthand of matrix may be convenient. 
%However, it is so short that I'm worried it may 
%collide with something else. I don't use at present.
%\newcommand{\m}[1]{\mathbf{#1}}
\newcommand{\adj}[0]{\mathrm{adj}}

%=====Theorem definitions=====
\newcounter{mytheoremorder}
\newtheorem{mydef}{Definition}
\newtheorem{myaxm}{Axiom}
\newtheorem{mythm}[mytheoremorder]{Theorem}
\newtheorem{myprop}[mytheoremorder]{Proposition}
\newtheorem{myex}{Example}

%=====Optimization====
\DeclareMathOperator*{\argmax}{arg\,max}
\DeclareMathOperator*{\argmin}{arg\,min}
\newcommand{\maximize}[0]{\mathrm{Maximize~}}
\newcommand{\minimize}[0]{\mathrm{Minimize~}}

%=====Probability====
\newcommand{\E}[0]{\mathbb{E}}
\newcommand{\var}[0]{\mathrm{Var}}
\newcommand{\cov}[0]{\mathrm{Cov}}

%=====Quick and Unified Reference====
%20120505
%Usage: \eq{\ref{xxx}}
%The reason I keep "\ref" away from definition, 
%and let user type it every time is that: 
%currently I'm working on Texmaker, and it can 
%trigger a selection panel when the sequence 
%"\ref" is found. Maybe further configuration of 
%texmake can make it do the same thing when the 
%following self-defined sequence is detected.
%This is left for future work. 
\newcommand{\req}[1]{\textbf{Eq~{#1}}}
\newcommand{\rfig}[1]{\textbf{Fig~{#1}}}
\newcommand{\rtbl}[1]{\textbf{Tbl~{#1}}}
\newcommand{\rpg}[1]{\textbf{P~{#1}}}
\newcommand{\rsec}[1]{\textbf{Section~{#1}}}


%This usually doesn't need modification 
\author{HU, Pili\thanks{hupili [at] ie [dot] cuhk [dot] edu [dot] hk}}

%Modify them accordingly===
\title{Expectation Maximization}
\date{April 15, 2012\thanks{Last compile:\today}}

\begin{document}

\maketitle
%>============================================
\begin{abstract}
	You're not reading the completed document. Only 
	brainstorm and some message from the author are there.  
\end{abstract}
%<=======Abstract ENd=========================

%>============================================
\pagebreak
\tableofcontents
\pagebreak
%<=======TOC ENd==============================



\section{Brainstorm}

Topics to becovered:
\begin{itemize}
	\item Collect how different people reach the $Q$ function, 
		like \cite{dempster1977em}\cite{bishop2006pattern}\cite{borman2004-emtut}. 
		Some are later interpretations, but worth to know. 
	\item Why do we choose $\ln$ in the $Q$ function? 
		Are there any other possibilities? I just find that any 
		concave function with $f(1)=0$ suffices in the derivation of 
		general form. The benefit of $\ln$ simply lies in the fact 
		that it favours the exponential family. 
	\item The analogy to variational inference approach. 
	\item An application of GMM. 
\end{itemize}






%>============================================
\section*{Acknowledgements}
\addcontentsline{toc}{section}{Acknowledgements}

%<=======Acknowledgements ENd=================

%>============================================
\addcontentsline{toc}{section}{References}
\input{../reference/gen_bib.bbl}
%<=======Bibliography ENd=====================

%>============================================
\section*{Appendix}
\addcontentsline{toc}{section}{Appendix}

\subsection*{Message From the Author}
The first time heard of EM algorithm was about 3 years ago. 
During all these years, I tried to learn and re-learn it 
for many times. Every time I find something new. Theoretists
and practitioners have different tastes in explaining things. 
The former makes rigorous derivations and provides 
different views. The latter constructs guide rules for 
people who only wants a solution but doesn't bother to 
derive. 

Recently, I eventually managed to derive the EM algorithm
in general and apply it to the GMM Maximum Likelihood learning 
problem. Although there are many tutorials out there, 
I determine to write my own, making it complete with 
theory, algorithm, and implementation. More importantly, 
I make the source publicly open. If people want to fix anything
or make their own notes upon this document, they
should find it convenient. 


%<=======Appendix ENd=========================

\end{document}
