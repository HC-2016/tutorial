\documentclass[11pt,a4paper]{article}
\usepackage[utf8x]{inputenc}
\usepackage{ucs}
\usepackage{amsmath}
\usepackage{amsfonts}
\usepackage{amssymb}
\usepackage{url}

\author{HU, Pili}
\title{Factor Graph Tutorial}

\begin{document}

\maketitle

\begin{abstract}
	
\end{abstract}

\pagebreak
\tableofcontents
\pagebreak

\section{An Example}

\section{Discussions}

Since the development of factor graph is boosted in the past decade, 
different authors come up with different description of similar problems. 
Not to distinguish right from wrong, I just regard those stuffs out there
as inconsistent. My opinion on some parts of past literature:
\begin{itemize}
	\item In Bishop's book\cite{bishop2006pattern}, chapter 8.4.5, P411. 
	The example is not good. Actually, when talking about that probability 
	maximization problem, we should know "product" corresponds to product operator, 
	and "sum" corresponds to max operator. In this case, the maginalization 
	operation for a single varialbe is indeed the maximization for each 
	instancde of that variable. Using local marginalized function(max), we 
	can certainly get the global probability maximization point considering
	all variables. 
	\item As for Dynamic Factor Graph, the author of this paper do not advocate 
	the abuse of this term like an extension of factor graph. FG itself is able 
	to model system dynamics, as we've already seen in those examples above. 
	Other authors may use the term DFG
\cite{wang2011-dynamic}
\cite{mirowski2009dynamic}
	, but their DFG is application specific. 
	Those graphs are essentially FG. Not until we examine the physical meaning of 
	some factor nodes do we realize their "dynamic" property. 
\end{itemize}


\input{reference/gen_bib.bbl}

\end{document}